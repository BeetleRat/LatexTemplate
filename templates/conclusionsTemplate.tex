В ходе лабораторной работы были освоены средства моделирования случайных величин (СВ) с произвольным распределением на основе равномерного распределения.\n
Построены имитационные модели двух потоков, в котором длительность промежутков времени между поступлениями заявок имеет показательный закон с параметрами $\lambda_1 = Ji1,\lambda_2 = Ji2$. Так же построена имитационная модель аддитивного потока, в котором длительность промежутков времени между поступлениями заявок имеет показательный закон с параметрами $\lambda_2 = Ji3$.\n 
Найдены выборочное среднее, выборочная дисперсия и выборочное среднеквадратическое отклонение параметра $\bar\lambda$:
\begin{align*}
&\bar\lambda_1=Ji1PractFull,\\
&D(\bar\lambda_1)=dispM1Full1\\
&\text{Среднеквадратическое отклонение }\bar\lambda_1=avgSqrt1Full1.
\end{align*}\n
Среднеквадратичное отклонение мало, значит можно положительно судить о значимости модели.\n
Так же разница между теоретическим и практическим значением $\lambda_1$ невысока $|\lambda_1 - \bar\lambda_1| = |Ji1 - Ji1PractFull| = Ji1error$, что так же дает положительно судить о значимости модели.
\begin{align*}
&\bar\lambda_2=Ji2PractFull,\\
&D(\bar\lambda_2)=dispM2Full1\\
&\text{Среднеквадратическое отклонение }\bar\lambda_2=avgSqrt2Full1.
\end{align*}\n
Среднеквадратичное отклонение мало, значит можно положительно судить о значимости модели.\n
Так же разница между теоретическим и практическим значением $\lambda_2$ невысока $|\lambda_2 - \bar\lambda_2| = |Ji2 - Ji2PractFull| = Ji2error$, что так же дает положительно судить о значимости модели.
\begin{align*}
&\bar\lambda_3=Ji3PractFull,\\
&D(\bar\lambda_3)=dispM3Full1\\
&\text{Среднеквадратическое отклонение }\bar\lambda_3=avgSqrt3Full1.
\end{align*}\n
Так же разница между теоретическим и практическим значением $\lambda_3$ невысока $|\lambda_1 + \lambda_2 - \bar\lambda| = |Ji1+Ji2-Ji3PractFull| = Ji3error$, что так же дает положительно судить о значимости модели.\n
Проверяем гипотезу по критерию Пирсона о распределении генеральной совокупности по закону Пуассона в условиях предоставленной выборки:
\begin{align*}
&\chi_\text{критическое}^2=XIcrit\\
&\chi_1^2=X1Full\\
&\chi_1^2<\chi_\text{критическое}^2\text{, гипотеза о Пуассоновости потока не отвергается}\\
&\chi_2^2=X2Full\\
&\chi_2^2<\chi_\text{критическое}^2\text{, гипотеза о Пуассоновости потока не отвергается}
\end{align*}\n
Осуществлена проверка статистической гипотезы о соблюдении свойства аддитивности пуассоновского потока (сумма пуассоновских потоков есть поток пуассоновский). Полученные статистические значения верны только для данной выборки.